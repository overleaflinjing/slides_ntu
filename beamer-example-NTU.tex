\documentclass[t]{beamer}

\usetheme[plain]{NTU}

\begin{document}

\title{Beamer Sample for NTU}
\subtitle{Based on Beamer version 3.07}
\author{Keck-Voon LING\\ ekvling@ntu.edu.sg}
\institute[NTU]{School of Electrical and Electronic Engineering\\ NTU}
\date{\today}

\begin{frame} %[plain]
  \titlepage
\end{frame}

\begin{frame}[c] %override the [t] option
  \frametitle{Itemized List}
  \begin{itemize}
    \item This is item 1
    \item This is item 2
  \end{itemize}
\end{frame}

\begin{frame}
	\frametitle{One Item at a Time}
	\begin{itemize}
		\item<1->One good argument
		\item<2->Another good argument, after one click
		\item<3->Last one, after another click
	\end{itemize}
\end{frame}

\begin{frame}
	\frametitle{A Slight Variations}
	This text will stay on all pages.
	\only<1>{
		\begin{itemize}
			\item<1->This will only appear on the \alert{first page}
			\item<1->This is also only for the \alert{first page}
		\end{itemize}
	}
	\only<2>{
		\begin{itemize}
			\item<2->This will only appear on the \alert{second page}
		\item<2->This is also only for the \alert{second page}
		\end{itemize}
      }
\end{frame}

\begin{frame}{Make Titles Informative.}

  You can create overlays\dots
  \begin{itemize}
  \item using the \texttt{pause} command:
    \begin{itemize}
    \item
      First item.
      \pause
    \item
      Second item.
    \end{itemize}
  \item
    using overlay specifications:
    \begin{itemize}
    \item<3->
      First item.
    \item<4->
      Second item.
    \end{itemize}
  \item
    using the general \texttt{uncover} command:
    \begin{itemize}
      \uncover<5->{\item
        First item.}
      \uncover<6->{\item
        Second item.}
    \end{itemize}
  \end{itemize}
\end{frame}

\begin{frame}[t]
  \frametitle{Two Columns}
  \begin{columns}
    \column{.5\textwidth}
    \begin{block}{Answered Questions}
      How many primes are there?
    \end{block}
    \column{.5\textwidth}
    \begin{block}{Open Questions}
      Is every even number the sum of two primes?
    \end{block}
  \end{columns}
\end{frame}

\begin{frame}[fragile]
\frametitle{Verbatim for Program Listing}
An Algorithm For Finding Primes Numbers.

\begin{verbatim}
int main (void)
{
  std::vector<bool> is_prime (100, true);
  for (int i = 2; i < 100; i++)
  if (is_prime[i])
  {
    std::cout << i << " ";

    for (int j = i; j < 100; is_prime [j] = false, j+=i);
  }
  return 0;
}
\end{verbatim}
\end{frame}

\begin{frame}[t]
  \frametitle{Uncovering a Formula Line-by-line}
  \begin{align}
    A &= B \\
    \uncover<2->{&= C \\}
    \uncover<3->{&= D \\}
    \notag
  \end{align}
  \vskip-1.5em
\begin{uncoverenv}<4>
Note that an  empty line is added without a tag and then insert a negative vertical skip
to undo the last line. See source for details.
\end{uncoverenv}

\end{frame}

\end{document}